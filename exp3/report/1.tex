\title{数据库专题训练——第二次作业
}
\author{MaskRay}
\maketitle

\section{任务}

实现近似抽取,近似串有两种度量方式:Levenshtein edit distance和Jaccard index。
需要实现\texttt{AEE}类的\texttt{createIndex}、\texttt{aeeED}、\texttt{aeeJaccard}三个方法。

\section{Faerie algorithm}

实现采用\cite{faerie}中提到的Faerie algorithm。

\subsection{索引}

读入描述entity的文件采用\texttt{mmap}:

\begin{minted}{c++}
struct stat s;
int fd = open(filename, O_RDONLY);
fstat(fd, &s);
mmap(0, s.st_size, PROT_READ, MAP_PRIVATE | MAP_POPULATE, fd, 0);
\end{minted}

其中的\texttt{MAP\_POPULATE}是为了让产生read-ahead的效果,加速文件读入。

索引的数据结构包含:

\begin{itemize}
  \item entity向量\texttt{e}
  \item 把Q-gram映射为entity编号的若干倒排索引\texttt{s2gid}。
  \item 若干倒排索引\texttt{gid2e},为每个entity中出现的Q-gram保存包含该Q-gram的entity向量。
\end{itemize}

对于每个输入的字串,使用类似Rabin-Karp string search算法的方式,
获取当前长为$q$的窗口中的Q-gram,计算出散列值,找到对应的倒排索引,
并在该索引末端加入当前行号。如果输入字串有重复的Q-gram,那么当前行号可能会在倒排索引中插入多次。
然后窗口向右移动一格,把之前的散列值通过rolling hash计算出下一时刻的值。

\texttt{s2gid}的每个倒排索引末尾添加哨兵元素\texttt{INT\_MAX}。

\subsection{近似抽取}

Levenshtein edit distance和Jaccard index均可用Faerie算法处理。
对于每个文档执行一次single-heap based algorithm。

\subsubsection{建立binary heap}

首先把长度为$n$的\texttt{doc}划分为$n-q+1$的Q-gram,经\texttt{s2gid}映射后得到$n-q+1$个指向倒排索引的指针。
以指针指向的值(entity编号)为第一关键字,位置为第二关键字使用Floyd's algorithm在$\Theta(|L|)$时间内建立binary heap \texttt{L}。

\subsubsection{Tournament sort进行N-way merge得到候选子串}

令$|e|$表示entity $e$的长度(和论文中定义不同)。

如果是Levenshtein edit distance查询,则设置$\bot_e=|e|-q+1-\tau$,
$\top_e=|e|-q+1+\tau$,$T_l=|e|-q+1-q\tau$;
如果是Jaccard index查询,则设置$\bot_e=\lceil\frac{|e|-q+1}{\tau}\rceil$,
$\top_e=\lfloor\frac{|e|-q+1}{\tau}\rfloor$,$T_l=\bot_e$。

\begin{enumerate}
  \item 记堆顶元素为\texttt{eeid},弹出所有等于\texttt{eeid}的元素得到\cite{faerie}中的$P_e$。
  \item 由于位置为第二关键字,得到的$P_e$以及有序了。
  \item 使用尺取法寻找包含$P_e$中至少$T_l$个元素,且包含的Q-gram数在区间$[\bot_e,\top_e]$内的所有窗口。
  \item 每个窗口可以产生若干候选子串。
\end{enumerate}

代码使用了batch-count pruning,使用尺取法时可以用\cite{faerie}中提到的二分检索方法,一次把指针移动若干步。

\subsubsection{Verification}

之后对每个候选子串进行检验,是否满足Jaccard index或Levenshtein编辑距离的阈值要求。

\paragraph{Levenshtein edit distance}

两个字符串的Levenshtein edit distance可以使用$\Theta(nm)$的Needleman–Wunsch algorithm计算。

注意到代码中使用到编辑距离的地方都有阈值$\tau$的限制,如果编辑距离超过阈值,那么它的实际值无关紧要。因此我们可以只计算动态规划矩阵中对角线带状区域的状态。

另外当其中某个字符串的长度小于等于128时,可以使用内置类型\texttt{\_\_int128}采用bit vector的算法\cite{edit03}加速到$\Theta(n) $。

两个字符串的Levenshtein edit distance的下界是字符集所有字符出现频数差的绝对值的和的一半,
当该下界小于等于阈值时再进行实际计算。可以用\texttt{\_mm\_sad\_epu8}来估算频数差的绝对值的和。
当某个字符频数差大于等于256时会低估实际值,但对结果没有影响。

\subsubsection{Jaccard index}

计算两个串$a$和$b$的Jaccard index时使用公式$\frac{\mathrm{overlap}}{|a|+|b|-\mathrm{overlap}}$。

采用scan count的方式计算$\mathrm{overlap}$。根据\texttt{s2gid}计算查询串所有Q-gram的编号,在计数容器中增加一。
然后对于候选串的所有Q-gram,若计数容器中的值大于零则减去零并加到答案中。再遍历候选串的所有Q-gram,把减去的值再加回来。

\section{Trie-based\cite{taste}}

令Levenshtein编辑距离阈值为$\tau$,对于每个entity $e$,划分为$\tau+1$个片段存入trie。

若文档$d$和某个$e$相似,则文档必然以$e$的某个片段$e_m$为子串。检索trie即可得到所有候选entity。

当探测到文档的某子串$d[i,j)$对应的trie节点包含某个entity $e$,则$e$就是一个候选entity,且可以知道
$e$的某个$e_m$已经出现在$d[i,j)$中了,包含$d[i,j)$的某个更长子串$d[k,l)$和$e$的编辑距离是:

\begin{eqnarray*}
\text{ED}(e,d[k,l)) &=& \text{ED}(e_l,d[k,i)) + \text{ED}(e_m,d[i,j)) + \text{ED}(e_r,d[j,k)) \\
&=& \text{ED}(e_l,d[k,i)) + \text{ED}(e_r,d[j,k))
\end{eqnarray*}

即需要把$e_m$前面的部分$e_l$和$d[*,i)$匹配,$e_m$后面的部分$e_r$和$d[j,*)$匹配,两部分可以分开考虑。
总的编辑距离就是两部分匹配结果编辑距离的和。如果左侧的$\text{ED}(e_l,d[k,i))=u\leq \tau$,则右侧的$\text{ED}(e_r,d[j,k))$
最大可以是$\tau-u$。

\paragraph{$\text{ED}(e_r,d[j,k))$}

$\text{ED}(e_r,d[j,k))$可以用带阈值限制的编辑距离算法计算。

\paragraph{多个串共同包含$e_m$时复用计算结果}

多个串共同包含$e_m$时,它们的右侧部分$e_r$可能也有公共前缀,这部分串的编辑距离结果可以复用。

把trie节点中检索到的entity按$e_r$组织为下层trie(原trie为上层trie)。求出下层trie的最左叶节点和$d[j,l)$的编辑距离后,
要切换到次左叶节点,它们的longest common prefix和$d[j,l)$的编辑距离动态规划值可以复用。

构建下层trie比较耗空间,可以借鉴suffix array的思路,把索引到的entity的右侧部分$e_r$排序,先计算字典需最小$e_r$和$d[j,l)$的编辑距离,
再计算字典序次小的$e_r$和$d[j,k)$的编辑距离,如此处理其他所有$e_r$。

另外如果发现计算到$d[g,*)$时发现所有$d[g,*)$都超过$\tau$了,那么可以立刻停止计算。

\paragraph{生成结果}

把匹配某个$e$左侧部分$e_l$的$d[k,i)$和匹配对应右侧部分$e_r$的$d[j,l)$求cartesian product,
如果两部分编辑距离和小于$\tau$,拼接两部分以及$e_m$可以得到一个结果子串。

文档的某个子串和某个$e$的公共部分可能较长,导致$e$的多个片段都在trie中被检索到。
产生的结果可能会重复,另外编辑距离的计算各片段间也会有很大的重叠。我没想到怎么利用。

\paragraph{Aho-Corasick automaton}

目前的方式是对于文档的每个位置都对上层trie探测得到候选entity,可以Aho-Corasick automaton改进。
注意Aho-Corasick automaton和Aho-Corasick algorithm的區別,前者會把trie改造爲圖(一些網絡資料稱之爲trie圖)。
探測的文檔子串的右邊界在不斷增加,左邊界要根據Aho-Corasick狀態的轉移選擇性變化。

\begin{minted}{c++}
TrieNode *p = trie_root, *q;
for (int i, j = 0; j < m; ) {
  u8 c = doc[j++];
  i += p->dep+1-p->ch[c]->dep;
  p = p->ch[c];
  for (ii = i, q = p; ; ) {
    // 考慮文檔d[ii,j)和節點q表示的e_m的下層trie q->inferior
    if (p->pi) {
      ii += p->depth-p->pi->depth;
      p = p->pi;
    } else
      break;
  }
}
\end{minted}

對於每個匹配的\texttt{p}都要沿着\texttt{p->pi->pi->\ldots}鍊上溯到根。可以對於每個trie節點記錄其最長的表示一個$e_m$的後綴(即\texttt{pi}的若干次冪)。
這樣可以被多串匹配部分的複雜度從$O(T+K+PK)$優化到$O(T+K)$,其中$T$爲文檔長度,$K$爲$e_m$出現次數和,$P$爲單個$e_m$的長度。

\section{演化}

Faerie 7.263
Faerie+ED共用 4.197
Trie-based初版 2.4s
使用Aho-Corasick automaton后2.264s
利用公共前缀后 2.133s

\printbibliography
