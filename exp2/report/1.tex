\title{数据库专题训练——第二次作业
}
\author{MaskRay}
\maketitle

\section{任务}

实现similarity join,近似串有两种度量方式:Levenshtein edit distance和Jaccard index。
需要实现\texttt{SimJoiner}类的\texttt{joinED}、\texttt{joinJaccard}两个方法。

\section{BruteForce}

枚举$S$集合的所有字符串,和$R$集合的所有字符串一一进行Levenshtein edit distance或Jaccard index的计算,
若满足阈值条件则添加到存放结果的向量,实现复杂度为$O(|R|\cdot |S|\cdot N)$,其中$N$为串的最大长度。

\section{DivideSkip}

执行$|S|$次similarity search。
采用filter-and-verification framework,代码中实现了索引、Filter和Verification三个部分。

\subsection{索引}

索引的数据结构是若干倒排索引,另外有一个hash table为Q-gram到倒排索引的映射。

执行\texttt{createIndex}构建索引时,
对于每个输入的字串,使用类似Rabin-Karp string search算法的方式,
获取当前长为$q$的窗口中的Q-gram,计算出散列值,找到对应的倒排索引,
并在该索引末端加入当前行号。
如果输入字串有重复的Q-gram,那么当前行号可能会在倒排索引中插入多次。
然后窗口向右移动一格,把之前的散列值通过rolling hash计算出下一时刻的值。

输入文件读入完毕后,对于hash table中的每个倒排索引(必然是非空的),
在末端加入作为哨兵元素的无穷大。

\subsection{Filter}

对于一个查询,使用类似构建索引时的Rabin-Karp string search algorithm,
对于每个长为$q$的窗口中的Q-gram,找到对应的倒排索引,
把指向首元素的指针放入列表$L$中。
设查询字串长度位$n$,那么$L$中将会有$n-q+1$个指针(可能重复)。

\subsubsection{建立binary heap}

列表$L$的元素是指向倒排索引的指针,以指针指向的值为关键字使用Floyd's algorithm
在$\Theta(|L|)$时间内建立binary heap $L$。

\subsubsection{Tournament sort进行N-way merge}

\begin{enumerate}
  \item 堆顶指针指向的值为$\mathrm{old}$,若old为无穷大则返回
  \item 计算堆顶元素指向的值$\mathrm{old}$的出现次数,若大于等于阈值$\mathit{overlap}$则添加到候选集中
  \item 把堆顶元素指针向前移动一格(即指向对应的倒排索引的下一项),若下一项仍等于原来指向的值则继续移动
  \item 上述操作后堆顶指针指向的值变大了,对它进行下滤操作
  \item 若新的堆顶指向的值等于$\mathrm{old}$则跳转到步骤3
\end{enumerate}

之后对候选集的每个元素进行检验,是否满足Jaccard index或Levenshtein编辑距离的阈值要求,
输出筛选后的。

\begin{enumerate}
  \item 使用启发函数把倒排索引划分为长索引和短索引两类,长索引有$\mathit{nlong}<\mathit{n}$个,其中$n$为索引个数
  \item 对短索引采用MergeSkip算法,对于短索引中所有出现次数不少于$\mathit{overlap}-\mathit{nlong}$的元素,在所有长索引中二分检索
  \item 将出现次数不少于$\mathit{overlap}$的元素添加到候选集中
  \item 把堆顶元素指针向前移动一格(即指向对应的倒排索引的下一项),若下一项仍等于原来指向的值则继续移动
  \item 上述操作后堆顶指针指向的值变大了,对它进行下滤操作
  \item 若新的堆顶指向的值等于$\mathrm{old}$则跳转到步骤4
  \item 从堆中弹出$\mathit{overlap}-\mathit{nlong}-1$个元素,使用二分检索把这些倒排索引指针移动至大于当前堆顶指向的值,再重新插入堆中
\end{enumerate}

\subsection{Verification}

\subsubsection{Levenshtein edit distance}

需要对每个候选字串和查询字串计算编辑距离。

两个字符串的Levenshtein edit distance可以使用$\Theta(nm)$的Needleman–Wunsch algorithm计算。

注意到代码中使用到编辑距离的地方都有阈值$\mathit{th}$限制,
如果编辑距离超过阈值,那么它的实际值无关紧要。
因此我们可以只计算动态规划矩阵中对角线带状区域的状态。

另外当其中某个字符串的长度小于等于128时,还可以采用bit vector的算法\cite{edit03}加速到$\Theta(n) $。

两个字符串的Levenshtein edit distance的下界是字符集所有字符出现频数差的绝对值的和的一半,
当该下界小于等于阈值时再进行实际计算。可以用\texttt{\_mm\_sad\_epu8}来估算频数差的绝对值的和。
当某个字符频数差大于等于256时会低估实际值,但对结果没有影响。

\subsubsection{Jaccard index}

采用scan count的方式。
先用rolling hash计算查询串所有Q-gram的标号,在计数容器中增加一。
然后对于候选串的所有Q-gram,若计数容器中的值大于零则减去零并加到答案中。
再便利候选串的所有Q-gram,把减去的值再加回来。

\section{ppjoin\cite{ppjoin}}

之前一个疑惑是没理解similarity join和执行若干次similarity search有什么区别。
原来是similarity search构建的索引能支持阈值不同的查询,而similarity join直接指定了固定阈值,
因此构建的索引可以利用这个性质。

\subsection{Prefix filtering}

按照一个全序排序entity包含的gram,那么若两个串的overlap满足:$O(x,y)\geq \alpha$,则$x$的前$|x|-\alpha+1$个q-gram
必然与$y$的前$|y|-\alpha+1$个gram有交集。

\paragraph{Jaccard index转化为overlap}

由$J(x,y)\geq t$可得$O(x,y)\geq \frac{t}{1+t}(|x|+|y|)$。

另由$ \frac{|x|}{|y|} \geq \frac{O(x,y)}{|x|+|y|-O(x,y)} \geq J(x,y) \geq t $
得$|x| \geq t |y|$

如果需要能所探寻到所有overlap超过$t|x|$的的串,取前缀长度为$|x|-\lceil t |x|\rceil+1$即可。

论文中没有提及某gram被同一个entity包含多次的情况。这时$x$的前缀和$y$的前缀的交集大小似乎不能有效地精确计算出来。

\subsection{ppjoin}

对于查询串$x$,计算$\text{pref}_x=\lceil t |x|\rceil+1$,探测前$\text{pref}_x$个gram的倒排索引后
得到一些候选entity,并且知道每个候选entity $y$的前$\text{pref}_y$个gram和$x$前$\text{pref}_x$个gram的交集大小,
如果加上$\min{|x|-\text{pref}_x,|y|-\text{pref}_y}$后不到下界$\alpha$,则$y$可以剔除。

需要一种类似数组的数据结构支持三种操作:清零、访问某一项、列出上次清零以来所有访问过的项。
可以用三个数组和一个计数器实现。

\subsection{Positional filtering}

根据overlap计算hamming distance上界后根据$x$和$y$的某几个对应的相等字符估算hamming distance下界,判断是否可行。

当gram有重复时比较麻烦,\texttt{SuffixFilterWeighted}中不能直接用\texttt{mid}作为$y$的枢轴,需要取
$y$中第一个取值和\texttt{mid}相同的字符。

对于可能重复的gram,另一种处理方式是让各gram取不同的值。

\printbibliography
