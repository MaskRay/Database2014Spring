\documentclass[11pt,a4paper]{article}

\usepackage{fontspec,verbatim,minted}
\usepackage{titlesec, titletoc}
\usepackage[titletoc]{appendix}
\usepackage{amsmath}
\usepackage[hyperfootnotes=false,colorlinks,linkcolor=blue,anchorcolor=blue,citecolor=blue]{hyperref}
\usepackage[backend=bibtex]{biblatex}
\defbibheading{bibliography}{\section{References}}
\bibliography{refs.bib}
\usepackage{subcaption}
\usepackage{enumitem}
\usepackage[slantfont,boldfont,CJKnumber]{xeCJK}
\usepackage{textgreek}

\usepackage{hyperref}
\usepackage{indentfirst}
\usepackage{float}			% don't automatically change location of figure [H]
\usepackage{chngpage}		% use \changetext to change page size
\usepackage{caption}\captionsetup{hypcap=true}  % ref to jump to object instead of caption
\setCJKmainfont{SimSun}
\setlength{\parindent}{2em}
\changetext{}{2.2cm}{-1.5cm}{-1.5cm}{}

\usepackage{hyperref}
\usepackage{indentfirst}
\usepackage{float}			% don't automatically change location of figure [H]
\usepackage{chngpage}		% use \changetext to change page size
\usepackage{caption}\captionsetup{hypcap=true}  % ref to jump to object instead of caption
\setlength{\parindent}{2em}
\changetext{}{2.2cm}{-1.5cm}{-1.5cm}{}

\usepackage{fancyhdr}
\pagestyle{fancy}
\lhead[]{}\rhead[]{}
\setlength{\headheight}{15.2pt}
\fancyhead[C]{\emph{数据库专题训练——第一次小作业
}}

\renewcommand{\today}{\number\year 年 \number\month 月 \number\day 日}
\renewcommand{\abstractname}{摘要}
\renewcommand{\contentsname}{目录}
\renewcommand{\tablename}{表}
\renewcommand{\figurename}{图}
\newcommand{\figref}[1]{\hyperref[fig:#1]{图\ref*{fig:#1}}}
\newcommand{\secref}[1]{\hyperref[sec:#1]{\ref*{sec:#1}节}}
\newcommand{\tabref}[1]{\hyperref[tab:#1]{表\ref*{tab:#1}}}

\def\ojoin{\setbox0=\hbox{$\Join$}%
\rule[0.1ex]{.27em}{.4pt}\llap{\rule[1.3ex]{.27em}{.4pt}}}
\def\leftouterjoin{\mathbin{\ojoin\mkern-5.8mu\Join}}
\def\rightouterjoin{\mathbin{\Join\mkern-5.8mu\ojoin}}
\def\fullouterjoin{\mathbin{\ojoin\mkern-5.8mu\Join\mkern-5.8mu\ojoin}}

\newlength\savewidth
\newcommand\shline{\noalign{\global\savewidth\arrayrulewidth\global\arrayrulewidth 1pt}
                   \hline
                   \noalign{\global\arrayrulewidth\savewidth}}
\newcommand\VRule[1][\arrayrulewidth]{\vrule width #1}

%\titleformat{\section}{\centering\bf\Large}{\CJKnumber{\thesection}}{1em}{}

\usepackage{breqn}

\begin{document}

\title{数据库专题训练——第一次小作业
}
\author{MaskRay}
\maketitle

\section{任务}

实现近似串查询,近似串有两种度量方式:Levenshtein edit distance和Jaccard index。
需要实现\texttt{SimSeacher}类的\texttt{createIndex}、\texttt{searchED}、\texttt{searchJaccard}三个方法。

代码中实现了brute-force、tournament sort、MergeSkip\cite{LLL08}和DivideSkip\cite{LLL08}四种查询算法。

\section{BruteForce}

枚举所有字符串,和查询字符串一一进行Levenshtein edit distance或Jaccard index的计算,
若满足阈值条件则添加到存放结果的向量。

\section{Tournament sort}

采用filter-and-verification framework,代码中实现了索引、Filter和Verification三个部分。

\subsection{索引}

索引的数据结构是若干倒排索引,另外有一个hash table为Q-gram到倒排索引的映射。

执行\texttt{createIndex}构建索引时,
对于每个输入的字串,使用类似Rabin-Karp string search算法的方式,
获取当前长为$q$的窗口中的Q-gram,计算出散列值,找到对应的倒排索引,
并在该索引末端加入当前行号。
如果输入字串有重复的Q-gram,那么当前行号可能会在倒排索引中插入多次。
然后窗口向右移动一格,把之前的散列值通过rolling hash计算出下一时刻的值。

输入文件读入完毕后,对于hash table中的每个倒排索引(必然是非空的),
在末端加入作为哨兵元素的无穷大。

\subsection{Filter}

对于一个查询,使用类似构建索引时的Rabin-Karp string search algorithm,
对于每个长为$q$的窗口中的Q-gram,找到对应的倒排索引,
把指向首元素的指针放入列表$L$中。
设查询字串长度位$n$,那么$L$中将会有$n-q+1$个指针(可能重复)。

\subsubsection{建立binary heap}

列表$L$的元素是指向倒排索引的指针,以指针指向的值为关键字使用Floyd's algorithm
在$\Theta(|L|)$时间内建立binary heap $L$。

\subsubsection{Tournament sort进行N-way merge}

\begin{enumerate}
  \item 堆顶指针指向的值为$\mathrm{old}$,若old为无穷大则返回
  \item 计算堆顶元素指向的值$\mathrm{old}$的出现次数,若大于等于阈值$\mathit{overlap}$则添加到候选集中
  \item 把堆顶元素指针向前移动一格(即指向对应的倒排索引的下一项),若下一项仍等于原来指向的值则继续移动
  \item 上述操作后堆顶指针指向的值变大了,对它进行下滤操作
  \item 若新的堆顶指向的值等于$\mathrm{old}$则跳转到步骤3
\end{enumerate}

之后对候选集的每个元素进行检验,是否满足Jaccard index或Levenshtein编辑距离的阈值要求,
输出筛选后的。

\subsection{Verification}

\subsubsection{Levenshtein edit distance}

对于\texttt{searchED}操作,需要对每个候选字串和查询字串计算编辑距离。

两个字符串的Levenshtein edit distance可以使用$\Theta(nm)$的Needleman–Wunsch algorithm计算。

注意到代码中使用到编辑距离的地方都有阈值$\mathit{th}$限制,
如果编辑距离超过阈值,那么它的实际值无关紧要。
因此我们可以只计算动态规划矩阵中对角线带状区域的状态。

另外当其中某个字符串的长度小于等于128时,还可以采用bit vector的算法\cite{edit03}加速到$\Theta(n) $。

\subsubsection{Jaccard index}

采用scan count的方式。
先用rolling hash计算查询串所有Q-gram的标号,在计数容器中增加一。
然后对于候选串的所有Q-gram,若计数容器中的值大于零则减去零并加到答案中。
再便利候选串的所有Q-gram,把减去的值再加回来。

\section{MergeSkip}

和tournament sort基本结构相同,对Filter部分做了优化,
即优化了统计每个字串的出现次数,过程如下:

\begin{enumerate}
  \item 堆顶指针指向的值为$\mathrm{old}$,若old为无穷大则返回
  \item 计算堆顶元素指向的值$\mathrm{old}$的出现次数,若大于等于阈值$\mathit{overlap}$则添加到候选集中
  \item 把堆顶元素指针向前移动一格(即指向对应的倒排索引的下一项),若下一项仍等于原来指向的值则继续移动
  \item 上述操作后堆顶指针指向的值变大了,对它进行下滤操作
  \item 若新的堆顶指向的值等于$\mathrm{old}$则跳转到步骤3
  \item 从堆中弹出$\mathit{overlap}-1$个元素,使用二分检索把这些倒排索引指针移动至大于当前堆顶指向的值,再重新插入堆中
\end{enumerate}

\section{DivideSkip}

结合了MergeSkip和MergeOut。

\begin{enumerate}
  \item 使用启发函数把倒排索引划分为长索引和短索引两类,长索引有$\mathit{nlong}<\mathit{n}$个,其中$n$为索引个数
  \item 对短索引采用MergeSkip算法,对于短索引中所有出现次数不少于$\mathit{overlap}-\mathit{nlong}$的元素,在所有长索引中二分检索
  \item 将出现次数不少于$\mathit{overlap}$的元素添加到候选集中
  \item 把堆顶元素指针向前移动一格(即指向对应的倒排索引的下一项),若下一项仍等于原来指向的值则继续移动
  \item 上述操作后堆顶指针指向的值变大了,对它进行下滤操作
  \item 若新的堆顶指向的值等于$\mathrm{old}$则跳转到步骤4
  \item 从堆中弹出$\mathit{overlap}-\mathit{nlong}-1$个元素,使用二分检索把这些倒排索引指针移动至大于当前堆顶指向的值,再重新插入堆中
\end{enumerate}

\section{其他}

Tournament类使用了curiously recurring template pattern。

\printbibliography


\end{document}
